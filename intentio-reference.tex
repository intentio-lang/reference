% !TeX root = ./intentio-reference.tex

\documentclass[12pt,a4paper,oneside,fleqn]{book}

\usepackage{amsmath}
\usepackage[T1]{fontenc}
\usepackage{graphicx}
\usepackage{geometry}
\usepackage{hyperref}
\usepackage[utf8]{inputenc}
\usepackage{listings}
\usepackage{makeidx}
\usepackage{mathtools}
\usepackage{url}
\usepackage{xcolor}

\usepackage{intentio-macros}

\setlength{\parindent}{0em}
\setlength{\parskip}{1em}

\hypersetup{
  colorlinks=true,
  linkcolor=blue,
  citecolor=red,
  urlcolor=cyan,
  pdfpagemode=UseOutlines,
  pdftitle={The Intentio Language Reference},
  pdfauthor={Anna Bukowska, Marek Kaput},
  bookmarksopen=true
}

\lstdefinestyle{stdstyle}{
  basicstyle=\ttfamily,
  breakatwhitespace=false,
  breaklines=true,
  keepspaces=true,
  showspaces=false,
  showstringspaces=false,
  showtabs=false,
  tabsize=2
}

\lstset{style=stdstyle}

\makeindex

\author{Anna Bukowska, Marek Kaput}

\title{The Intentio Language Reference}

\begin{document}

\frontmatter

\maketitle

\tableofcontents

\chapter{Preface}

We live in times of rapid emerging of new, modern programming languages. Some of them, like Rust\cite{TheRustProgrammingLanguage}, Go\cite{TheGoProgrammingLanguage} or Swift\cite{TheSwiftProgrammingLanguage}, have proved that programming language styles have not settled down and there is still room for new ideas, especially for merging existing paradigms. Fundamentally, one can observe a shift from imperative programming to functional programming.

Despite all these changes, not all ideas get a chance to shine. Some of them are becoming forgotten and treated as esoteric. One of these is goal-oriented evaluation, with Icon\cite{TheIconProgrammingLanguage} being one of its most \emph{iconic} implementers. Authors of this document believe that Icon exposes some very interesting ideas and they made an attempt to recreate them in a new programming language: \emph{Intentio}, named after Latin for \emph{intention}.

This document is the primary reference for the Intentio programming language. It \emph{semi-formally}\footnote{This document tries to maintain reasonably formal description of all items, but there are no guarantees all cases are described. As a fallback, the \emph{intentioc}\cite{intentioc} compiler can be used as secondary reference.} describes each language construct and its use.
 
This document does not serve as a beginner-friendly introduction to the language.

\section*{Goals}

The primary goals when designing Intentio was for language to satisfy following constraints:
\begin{enumerate}
  \item It should feature core concept of Icon language: goal-oriented evaluation
  \item It should support Unicode character set
  \item It should be fast and easy to build prototype applications, the language should be \emph{ergonomic}\footnote{Source code of Intentio programs should be concise, pleasant to write and easy to reason about.} from developer perspective \textbf{and} \emph{IDE friendly}\footnote{Source code of Intentio programs should be easily processable by text editors and analysis tools.}
  \item It should feature rich capabilities in processing textual data
\end{enumerate}

Our main goals were \textbf{not}:
\begin{enumerate}
  \item The language should be general purpose language
  \item Compilation time should be short
  \item Memory usage of Intentio programs should be low
  \item It should be easy to integrate with other languages
\end{enumerate}

\section*{Acknowledges}

The structure and parts of content of this document are inspired by three existing language specifications which we believe are good examples to follow: The Rust Reference\cite{TheRustReference}, Haskell 2010 Language Report\cite{Marlow_haskell2010} and The Go Programming Language Specification\cite{TheGoProgrammingLanguage}.


\mainmatter

\part{The Intentio Language}
\chapter{Introduction}

Intentio is domain specific, imperative programming language oriented for processing textual data. Intentio provides goal-oriented execution, generators, strong dynamic typing with optional type annotations, and a rich set of primitive data types, including Unicode strings, lists, arrays, maps, sets, arbitrary and fixed precision integers, and floating-point numbers. Intentio tries to incorporate ideas of the Icon\cite{TheIconProgrammingLanguage} programming language into modern programming patterns.

This part defines the syntax of Intentio language and informal abstract semantics for the meaning of such programs. We leave as implementation detail how Intentio programs are manipulated, interpreted, compiled, etc. This includes all steps from source code to running program, programming environments and error messages. This also means that this document do not describe the reference compiler for Intentio language - \emph{intentioc}\cite{intentioc}.


\section{Notational Conventions}

\subsection*{Grammar}

Throughout this document a BNF-based notational syntax is used to describe lexical structure and grammar:

\begin{bnf}
  nonterminal  \eq \term{terminal} \gor \term{repeated}+ \\
  nonterminal' \eq nonterminal*
\end{bnf}

Following conventions are used for presenting productions syntax:

\begin{bnfutils}
\begin{center}
\begin{tabular}[t]{ll}
  \((pat)\)            & grouping \\
  \([pat]\)            & optional (0 or 1 times) \\
  \(pat*\)             & optional repetition (0 to \(n\) times) \\
  \(pat+\)             & repetition (1 to \(n\) times) \\
  \(pat_1 \ pat_2\)    & concatenation \\
  \(pat_1 \gor pat_2\) & alternation \\
  \(\term{token}\)     & terminal symbol (in fixed-width font)
\end{tabular}
\end{center}
\end{bnfutils}

\subsubsection*{Unicode productions}

A few productions in Intentio's grammar permit Unicode\cite{Unicode} code points outside the ASCII range. These productions are defined in terms of character properties specified in the Unicode standard, rather than in terms of ASCII-range code points. Intentio compilers are expected to make use of new versions of Unicode as they are made available.

\subsection*{Source code listings}

Examples of Intentio program fragments are given in fixed-width font:

\begin{lstlisting}
  fun main():
    x, y := 4, 3
    writeln(f"sum = ${x + y}")
\end{lstlisting}

In some situations, there are \emph{placeholders} in program fragments representing arbitrary pieces of Intentio code are written in italics. By convention $e$ will mean expressions, $d$ - item declarations, $t$ - types, etc.:

\begin{lstlisting}[mathescape=true]
  if $e_1$ { $e_2$ } else { $e_3$ }
\end{lstlisting}


\section{Compile-time and Run-time}

Intentio's semantics obey a \emph{phase distinction} between compile-time\index{compile-time} and run-time\index{run-time}\footnote{In interpreter environments, compile-time would consist of syntactic analysis and linting.}. Semantic rules that have a static interpretation govern the success or failure of compilation, while semantic rules that have a dynamic interpretation govern the behaviour of the program at run-time.


\section{Program Structure}

An Intentio program is structured syntactically and semantically into five abstract levels:

\begin{enumerate}
  \item At the topmost of each Intentio program or library is an \emph{assembly}. In compiled environments assembly is an unit of compilation, while in interpreted environments assembly is a whole set of loaded modules.
  \item At the topmost of each assembly is a set of \emph{modules}. Modules provide a way to control namespaces and to re-use software in larger programs. A particular source code file of Intentio program consists of one or more modules.
  \item The top level of each module is a set of \emph{item declarations}. An item is a component of module, such as a function, submodule or constant variable
  \item Items which contain the real, executable code are built of \emph{expressions}. Expression denotes how to evaluate a \emph{term} and evaluating expression returns a \emph{result}.
  \item At the bottom level is Intentio's \emph{lexical structure}. It describes how to build tokens - the most basic blocks of program's source code from sequences of characters in source file.
\end{enumerate}


\section{Values, Types, Terms and Results}

A \emph{value}\index{value} is a representation of some entity which can be manipulated by a program. A \emph{type}\index{type} is a tag that defines the interpretation of value representation. Values and types are not mixed in Intentio. Values by itself are untyped, value type is required to perform any kind of operation on value. Values in Intentio are always in normal form.

A \((value, type)\) pair is called a \emph{term}\index{term}. Terms represent data yielded from evaluating expressions. Intentio is \emph{strongly typed} so implicit type conversions do not exist in the language, but it is not prohibited to hide casts in callee (thus \lstinline{5 + 4.0} runs successfully).

Evaluating expressions may either succeed or fail. A tagged union of successfully evaluated result or failure with information describing what failed is called a \emph{result}\index{result}. Terms and results are the basic blocks of representing information in Intentio.

Following Haskell-style code listing describes relationships between these concepts:

\begin{lstlisting}[language=Haskell]
  data Value = ...
  data Type = ...

  newtype Term = (Value, Type)

  data Result = Succ Term
              | Fail Term
\end{lstlisting}


\section{Namespaces}

There are three distinct namespaces in Intentio:

\begin{description}
  \item [Item namespace\index{item namespace}] Consists item and variable names.
  \item [Module namespace\index{module namespace}] Consists of module names and import renames.
  \item [Type namespace\index{type namespace}] Consists of type names.
\end{description}

There are no constraints on names belonging to particular namespace, therefore it is possible for name \texttt{Int} to simultaneously denote an item/variable, module and type.

\chapter{Lexical structure}

This chapter describes the lexical structure of Intentio. Most of the details may be skipped in a first reading of the reference.

In this chapter all whitespace is expressed explicitly in syntax descriptions, there is no implicit space between juxtaposed symbols. Terminal characters represent real characters in program source code.

\section{Input format}

Intentio program source is interpreted as a sequence of Unicode code points encoded in UTF-8, though most grammar rules are defined in terms of printable ASCII code points.

Intentio is \emph{case sensitive} language and each code point is distinct; for instance, upper and lower case letters are different characters.

The NUL character (U+0000) may be not allowed in whole program source text.

If an UTF-8-encoded byte order mark (U+FEFF) is the first Unicode code point in program source text, it may be ignored. Byte order mark may be not allowed anywhere else in program source text.

\section{Special Lexical Productions}

Following productions define Unicode character sets which are used to define non pure ASCII productions. These productions do not have any semantical meaning themselves.

\begin{bnfutils}
\begin{itemize}
  \item \(\term{XID\_start}\) and \(\term{XID\_continue}\) are sets of characters that have propertes \emph{XID\_start} and \emph{XID\_continue} as in Unicode Standard Annex \#31\cite{UAX31}, these productions define valid identifier characters
  \item \(\term{non\_null}\) is any single Unicode character with all implementation-specific character restrictions applied
  \item \(\term{non\_eol}\) is \(\term{non\_null}\) restricted to exclude \texttt{\textbackslash n} (U+000A)
  \item \(\term{non\_double\_quote}\) is \(\term{non\_null}\) restricted to exclude \texttt{"} (U+0022)
\end{itemize}
\end{bnfutils}

\section{Identifiers, Keywords and Operators}

\begin{bnf}
  ident \eq \gdiff{( \ \term{XID\_start} \ (\term{XID\_continue} \gor \term{'})* \ )}{keyword} \\
  \\
  keyword \eq
             \term{abstract}
      \gor   \term{and}
      \gor   \term{break}
      \gor   \term{case}
      \gor   \term{const}
      \gor   \term{continue}
      \gorln \term{do}
      \gor   \term{else}
      \gor   \term{enum}
      \gor   \term{fun}
      \gor   \term{if}
      \gor   \term{impl}
      \gor   \term{import}
      \gor   \term{in}
      \gorln \term{is}
      \gor   \term{loop}
      \gor   \term{mod}
      \gor   \term{not}
      \gor   \term{or}
      \gor   \term{return}
      \gor   \term{struct}
      \gor   \term{type}
      \gorln \term{where}
      \gor   \term{while}
      \gor   \term{yield}
      \gor   \term{\_} \\
  \\
  op \eq TODO
\end{bnf}

An \emph{identifier} consits of a "letter" or underscore followed by zero or more letters, digits, underscores, and single quotes. \emph{Keywords} are identifier-like tokens which have special meaning in the grammar, all of them are excluded from the \(ident\) rule.

Implementations that offer lints or warnings for unused parameters/variables/items are encouraged to suppress such warnings for identifiers beginning with underscore. This allows programmers to use \texttt{\_arg} for a parameter that they expect to be unused.

\section{Paths}

\begin{bnf}
  path \eq \term{:}? \ ( ident \ \term{:} )* \ ident
\end{bnf}

TODO

\section{Comments}

\section{Numeric Literals}

\section{String Literals}

\chapter{Modules and Assemblies}

Similarly to items, modules are also entirely determined at compile-time, remain fixed during execution, and may reside in read-only memory. This limitation does not apply to assemblies (including single module assemblies). It is possible to provide mechanisms to dynamically compile, link, load and unload assemblies at run-time.

\chapter{Items}

This chapter describes syntax and informal semantics of Intentio items. An \emph{item}\index{item} is a component of a module. Items are uniquely named within module and are organized in flat structure.

Items are entirely determined at compile-time, remain fixed during execution, and may reside in read-only memory.

\begin{bnf}
  \gd{itemdecl} \eq    \gp{fundecl}    \gcomm{function}
                \gorln \gp{importdecl} \gcomm{import declaration}
\end{bnf}

\section{Functions}

\begin{bnf}
  \gd{fundecl} \eq \term{fun} \ \term{\gp{id}} \ \term{(} \ \gtry{\gp{funparams}} \ \term{)} \ \gp{braceblock} \\
  \gd{funparams} \eq \term{\gp{id}} \ \gmany{ \ \term{,} \ \term{\gp{id}} \ } \ \gtry{\term{,}}
\end{bnf}

A \emph{function}\index{item!function} is a named block, along with optional set of parameters. Functions are declared with the keyword \lstinline{fun}. Function returns the result of evaluating of contained block. Functions may declare a set of input variables as parameters, through which the caller passes arguments into the function. Input variables behave the same as normal variables defined within function block.

\begin{example}[Simple function]
\begin{lstlisting}
fun the_hardest_calculation(x) {
  x * 2
}
\end{lstlisting}
\end{example}

\section{Import declarations}

\begin{bnf}
  \gp{importdecl} \eq    \term{import} \ \term{\gp{id}} \gcomm{qualified import}
                  \gorln \term{import} \ \term{\gp{id}} \ \term{as} \ \term{\gp{id}} \gcomm{qualified renamed import}
\end{bnf}

An \emph{import declaration}\index{item!import declaration} creates local name binding in current scope, synonymous with specified module. Exported items of imported module can be accessed using \emph{qualified identifier} syntax.

Import declarations support two forms:

\begin{description}
  \item [Qualified import] \hfill \\
    Binding target module to its name: \lstinline{import a}
  \item [Qualified renamed import] \hfill \\
    Binding target module to new local name: \lstinline{import a as b}
\end{description}

An import declaration declares a dependency relation between the importing and imported module. Modules may be mutually recursive, though module must not import itself, directly or indirectly.

\begin{example}[Import declarations]
\begin{lstlisting}
import io
import math as m

fun main() {
  f = io:open("result.txt", "w");
  io:writeln(f, m:sin(m:pi));
}
\end{lstlisting}
\end{example}

\chapter{Expressions}

This chapter describes syntax and semantics of Intentio \emph{expressions}. Intentio is an expression language. This means that all forms of result-producing or effect-causing evaluation fall into uniform syntax category of expressions. Usually each kind of expression can nest within each other kind of expression, and rules for evaluation of expressions involve specifying both the result produced by the expression and the order in which its sub-expressions are themselves evaluated.

Intentio does not have a concept of \emph{statement} known from other programming languages.

\begin{bnf}
  \gd{expr} \eq    \term{\gp{qid}}                     \gcomm{variable or item}
            \gorln \term{\gp{literal}}
            \gorln blockexpr                           \gcomm{block expression}
            \gorln unopexpr                            \gcomm{unary operator expression}
            \gorln binopexpr                           \gcomm{binary operator expression}
            \gorln \term{(} \ \gp{expr} \ \term{)}     \gcomm{parenthesized expression}
            \gorln callexpr                            \gcomm{function call}
            \gorln loopexpr                            \gcomm{loops}
            \gorln ifexpr                              \gcomm{conditionals}
            \gorln returnexpr                          \gcomm{return expression}
\end{bnf}

An \emph{expression}\index{expression} is an syntactic construct that \emph{evaluates} to a term. It may either \emph{succeed} or \emph{fail} resulting in a result. During the evaluation, expression may perform \emph{side effects}, for example it can mutate some state or perform execution jump. The meaning of each kind of expression dictates several things:

\begin{itemize}
  \item Whether or not to evaluate the sub-expressions when evaluating the expression
  \item The order in which to evaluate the sub-expressions
  \item How to combine the sub-expressions' results to obtain the result of the expression
\end{itemize}


\part{The Intentio Standard Library}
\chapter{Introduction}

This part defines the Intentio Standard Library (shortly \emph{stdlib}), its contents, semantics and the \emph{prelude} which is automatically imported in each Intentio program. This library provides essentials for building proper Intentio programs, some of which should be used by the language implementation through compiler intrinsics. For developer convenience it also provides common utilities which ease application development and make a standard for code interoperation. The standard library must be distributed with each implementation of the Intentio language.


\appendix

\chapter{Influences}

Intentio is not particularly original language, having the language Icon as the main source of inspiration, but also borrowing design element from wide range of other sources. Some of these are listed below:

\begin{itemize}
 \item Icon\cite{TheIconProgrammingLanguage}: goal-oriented execution, generators
 \item Rust\cite{TheRustProgrammingLanguage}: syntax
 \item Erlang\cite{TheErlangProgrammingLanguage}: syntax
 \item Python\cite{ThePythonProgrammingLanguage}: syntax
\end{itemize}


\backmatter

\clearpage
\addcontentsline{toc}{chapter}{Bibliography}
\bibliographystyle{plain}
\bibliography{bibliography}

\clearpage
\addcontentsline{toc}{chapter}{Index}
\printindex

\end{document}
