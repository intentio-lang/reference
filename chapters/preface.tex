\chapter{Preface}

We live in times of rapid emerging of new, modern programming languages. Some of them, like Rust\cite{TheRustProgrammingLanguage}, Go\cite{TheGoProgrammingLanguage} or Swift\cite{TheSwiftProgrammingLanguage}, have proved that programming language styles have not settled down and there is still room for new ideas, especially for merging existing paradigms. Fundamentally, one can observe a shift from imperative programming to functional programming.

Despite all these changes, not all ideas get a chance to shine. Some of them are becoming forgotten and treated as esoteric. One of these is goal-oriented evaluation, with Icon\cite{TheIconProgrammingLanguage} being one of its most \emph{iconic} implementers. Authors of this document believe that Icon exposes some very interesting ideas and they made an attempt to recreate them in a new programming language: \emph{Intentio}, named after Latin for \emph{intention}.

This document is the primary reference for the Intentio programming language. It \emph{semi-formally}\footnote{This document tries to maintain reasonably formal description of all items, but there are no guarantees all cases are described. As a fallback, the \emph{intentioc}\cite{intentioc} compiler can be used as secondary reference.} describes each language construct and its use.
 
This document does not serve as a beginner-friendly introduction to the language.

\section*{Goals}

The primary goals when designing Intentio was for language to satisfy following constraints:
\begin{enumerate}
  \item It should feature core concept of Icon language: goal-oriented evaluation
  \item It should support Unicode character set
  \item It should be fast and easy to build prototype applications, the language should be \emph{ergonomic}\footnote{Source code of Intentio programs should be concise, pleasant to write and easy to reason about.} from developer perspective \textbf{and} \emph{IDE friendly}\footnote{Source code of Intentio programs should be easily processable by text editors and analysis tools.}
  \item It should feature rich capabilities in processing textual data
\end{enumerate}

Our main goals were \textbf{not}:
\begin{enumerate}
  \item The language should be general purpose language
  \item Compilation time should be short
  \item Memory usage of Intentio programs should be low
  \item It should be easy to integrate with other languages
\end{enumerate}

\section*{Acknowledges}

The structure and parts of content of this document are inspired by three existing language specifications which we believe are good examples to follow: The Rust Reference\cite{TheRustReference}, Haskell 2010 Language Report\cite{Marlow_haskell2010} and The Go Programming Language Specification\cite{TheGoProgrammingLanguage}.
