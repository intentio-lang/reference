\chapter{Lexical structure}

This chapter describes the lexical structure of Intentio. Most of the details may be skipped in a first reading of the reference.

In this chapter all white space is expressed explicitly in syntax descriptions, there is no implicit space between juxtaposed symbols. Terminal characters represent real characters in program source code.

\section{Input format}

Intentio program source is interpreted as a sequence of Unicode code points encoded in UTF-8, though most grammar rules are defined in terms of printable ASCII code points.

Token stream of Intentio program is defined as:

\begin{bnf}
  \gd{lexprogram} \eq \gtry{\gp{whitespace}} \ \gmany{ \ \gp{token} \ \gtry{\gp{whitespace}} \ } \\
  \gd{token}      \eq  \gp{id}
                  \gor \gp{qid}
                  \gor \gp{keyword}
                  \gor \gp{operator}
                  \gor \gp{literal}
\end{bnf}

Intentio is \emph{case sensitive} language and each code point is distinct; for instance, upper and lower case letters are different characters.

The NUL character (U+0000) may be not allowed in whole program source text.

If an UTF-8-encoded byte order mark (U+FEFF) is the first Unicode code point in program source text, it may be ignored. Byte order mark may be not allowed anywhere else in program source text.

\section{Special Lexical Productions}

Following productions define Unicode character sets which are used to define non-pure ASCII productions. These productions do not have any semantic meaning themselves.

\begin{bnfutils}
\begin{itemize}
  \item \(\gd{xidstart}\) and \(\gd{xidcont}\) are sets of characters that have properties \emph{XID\_start} and \emph{XID\_continue} as in Unicode Standard Annex \#31\cite{UAX31}, these productions define valid identifier characters
  \item \(\gd{any}\) is any single Unicode character with all implementation-specific character restrictions applied
  \item \(\gd{eol}\) matches either line feed character \texttt{\textbackslash n} (U+000A, Unix-style newline marker) or carriage return and then line feed characters \texttt{\textbackslash r\textbackslash n} (U+000D U+000A, Windows-style newline marker)
  \item \(\gd{whitespacechar}\) matches any character that has the \emph{Pattern\_White\_Space} Unicode property, namely:

    \begin{itemize}
      \item horizontal tab \texttt{\textbackslash t} (U+0009)
      \item line feed \texttt{\textbackslash n} (U+000A)
      \item vertical tab (U+000B)
      \item form feed (U+000C)
      \item carriage return \texttt{\textbackslash r} (U+000D)
      \item space (U+0020)
      \item next line (U+0085)
      \item left-to-right mark (U+200E)
      \item right-to-left mark (U+200F)
      \item line separator (U+2028)
      \item paragraph separator (U+2029)
    \end{itemize}
\end{itemize}
\end{bnfutils}

Additionally:

\begin{bnf}
  \gd{noneol} \eq \gdiff{\gp{any}}{\gp{eol}} \\
  \\
  \gd{decdig} \eq \term{0} \gor \term{1} \gor ... \gor \term{9} \\
  \gd{bindig} \eq \term{0} \gor \term{1} \\
  \gd{octdig} \eq \term{0} \gor \term{1} \gor ... \gor \term{7} \\
  \gd{hexdig} \eq  \term{0} \gor \term{1} \gor ... \gor \term{9}
              \gor \term{A} \gor \term{B} \gor ... \gor \term{F}
              \gor \term{a} \gor \term{b} \gor ... \gor \term{f}
\end{bnf}

\section{Identifiers, Keywords and Operators}

\begin{bnf}
  \gd{id}  \eq \gdiff{( \ \gp{xidstart} \ \gmany{\gp{xidcont} \gor \term{'}} \ )}{\gp{keyword}} \\
  \gd{qid} \eq \gp{id} \ \term{:} \ \gp{id} \\
  \\
  \gd{keyword} \eq
               \term{and}
        \gor   \term{break}
        \gor   \term{const}
        \gor   \term{do}
        \gor   \term{else}
        \gor   \term{enum}
        \gor   \term{export}
        \gorln \term{fail}
        \gor   \term{fun}
        \gor   \term{if}
        \gor   \term{impl}
        \gor   \term{import}
        \gor   \term{in}
        \gor   \term{is}
        \gor   \term{module}
        \gorln \term{none}
        \gor   \term{not}
        \gor   \term{or}
        \gor   \term{return}
        \gor   \term{struct}
        \gor   \term{succ}
        \gor   \term{trait}
        \gorln \term{type}
        \gor   \term{while}
        \gor   \term{xor}
        \gor   \term{yield}
        \gor   \term{\_} \\
  \\
  \gd{operator} \eq
                \term{+}
         \gor   \term{-}
         \gor   \term{*}
         \gor   \term{/}
         \gorln \term{(}
         \gor   \term{)}
         \gor   \term{[}
         \gor   \term{]}
         \gor   \term{\{}
         \gor   \term{\}}
         \gor   \term{;}
         \gorln \term{==}
         \gor   \term{<}
         \gor   \term{<=}
         \gor   \term{>}
         \gor   \term{>=}
         \gor   \term{===}
         \gor   \term{!==}
         \gorln \term{=}
\end{bnf}

An \emph{identifier}\index{identifier} consists of a "letter" or underscore followed by zero or more letters, digits, underscores, and single quotes. Simple, unqualified identifiers (\(id\)) are always resolved within current module and scope. In order to be able to specify which module identifier belongs to, in most places identifier may be prefixed with the module name and "\texttt{:}" character to form a \emph{qualified identifier}\index{identifier!qualified identifier} (\(qid\)).

\emph{Keywords}\index{keyword} are identifier-like tokens which have special meaning in the grammar, all of them are excluded from the \(id\) rule. \emph{Operators}\index{operator} are another special tokens, these ones are formed from symbol characters. \(keyword\) and \(operator\) productions have no use in Intentio grammar definition, instead, particular tokens are used.

Implementations that offer hints or warnings for unused parameters/variables/items are encouraged to suppress such warnings for identifiers beginning with an underscore. This allows programmers to use \texttt{\_arg} for a parameter that they expect to be unused.

\section{Comments and White space}

\begin{bnf}
  \gd{whitespace} \eq \gmanya{ \ \gp{whitespacechar} \gor \gp{comment} \ } \\
  \gd{comment}    \eq \term{\#} \ \gmany{\gp{noneol}} \ \gp{eol}
\end{bnf}

A \emph{white space}\index{white space} is a non-empty sequence of white space characters or \emph{comments}\index{comment}. Comments in Intentio are only line-based, there is no concept of block comment.

\section{Literals}

\begin{bnf}
  \gd{literal} \eq \gp{integer} \gor \gp{float} \gor \gp{string} \gor \term{none}
\end{bnf}

\subsection{Numeric Literals}

\begin{bnf}
  \gd{decimal}     \eq \gp{decdig} \ \gmany{ \ \gp{decdig} \gor \term{\_} \ } \\
  \gd{binary}      \eq (\term{0b} \gor \term{0B}) \ \gp{bindig} \ \gmany{ \ \gp{bindig} \gor \term{\_} \ } \\
  \gd{octal}       \eq (\term{0o} \gor \term{0O}) \ \gp{octdig} \ \gmany{ \ \gp{octdig} \gor \term{\_} \ } \\
  \gd{hexadecimal} \eq (\term{0x} \gor \term{0X}) \ \gp{hexdig} \ \gmany{ \ \gp{hexdig} \gor \term{\_} \ } \\
  \\
  \gd{integer} \eq \gp{binary} \gor \gp{octal} \gor \gp{decimal} \gor \gp{hexadecimal} \\
\end{bnf}

\begin{bnf}
  \gd{exponent} \eq ( \term{e} \gor \term{E} ) \ \gtry{ \term{+} \gor \term{-} } \ \gmany{\term{\_}} \ \gp{decimal} \\
  \\
  \gd{float} \eq    \gp{decimal} \ \term{.} \ \gp{decimal} \ \gtry{\gp{exponent}}
             \gorln \gp{decimal} \ \gp{exponent}
\end{bnf}

A \emph{numeric literal}\index{literal!numeric literal} is either an \emph{integer literal}\index{literal!numeric literal!integer literal} or \emph{floating-point literal}\index{literal!numeric literal!floating-point literal}. Integer literals may be given in decimal (the default), binary (prefixed by \texttt{0b} or \texttt{0B}), octal (prefixed by \texttt{0o} or \texttt{0O}) or hexadecimal (prefixed by \texttt{0x} or \texttt{0X}) notation. Floating-point literals are always decimal. A floating literal must contain digits both before and after the decimal point. A "\texttt{\_}" character is allowed inside numeric literals for visual separation of digit groups, for instance \lstinline{476_981__109_528_} is equal to \lstinline{476981109528}. Regardless of the input method, only the value is stored. This means that after number is printed, no underlines will be displayed on the screen. Negative numeric literals are described grammatically, not lexically.

\subsection{String Literals}

\begin{bnf}
  \gd{string} \eq \gtry{\gp{stringmod}} \ ( \ \gp{string'} \gor \gp{charstring} \gor \gp{rawstring} \gor \gp{regexstring} \ )
\end{bnf}

Intentio features very flexible string literals syntax. String literals can be written in four forms (regular, character, raw and regular expression) that make the literal compile to one of three value types (\lstinline{String}, \lstinline{Char} or \lstinline{Regex}). String literals can be also prefixed with few modifiers that alter literal value before compiling it to value.

\subsubsection*{Regular String Literals}

\begin{bnf}
  \gd{string'} \eq \term{"} \ \gmany{ \ \gdiff{\gp{any}}{(\term{"} \gor \term{\textbackslash})} \gor \gp{escseq} \ } \ \term{"}
\end{bnf}

A \emph{string literal}\index{literal!string literal} is a sequence of Unicode characters, typed either directly or via escape sequence. String literals are compiled to \lstinline{String} terms.

\subsubsection*{Character Literals}

\begin{bnf}
  \gd{charstring} \eq \term{c"} \ ( \ \gdiff{\gp{any}}{(\term{"} \gor \term{\textbackslash})} \gor \gp{escseq} \ ) \ \term{"}
\end{bnf}

A \emph{character literal}\index{literal!string literal!character literal} is a single Unicode character string, typed either directly or via escape sequence. Character literals are compiled to \lstinline{Char} terms. The implementation is required to verify that character literal makes only one character.

\subsubsection*{Raw String Literals}

\begin{bnf}
  \gd{rawstring}      \eq \term{r} \ \gp{rawstring'}(0) \\
  \gd{rawstring'}(n)  \eq    \term{"} \ \gp{rawstring''}(n) \ \term{"}
                      \gorln \term{\#} \ \gp{rawstring'}(n+1) \ \term{\#} \\
  \gd{rawstring''}(n) \eq \textnormal{An \(\gmany{\gp{any}}\) that does not contain} \\
                      & & \textnormal{\term{"} followed by \term{\#} repeated \(n\) times.}
\end{bnf}

A \emph{raw string literal}\index{literal!string literal!raw string literal} does not process any escape sequences. It starts with letter \texttt{r}, followed by zero or more repetitions of hash symbol (\texttt{\#}) and double quote (\texttt{"}). The raw string body can contain any sequence of Unicode characters and is terminated only by another double quote (\texttt{"}) followed by the same number of hashes (\texttt{\#}) that proceeded the opening quote.

\begin{example}[Raw string literals]
\begin{lstlisting}
"foo"      == r"foo"            # foo
"\"foo\""  == r#""foo""#        # "foo"
"x #\"# y" == r##"x #"# y"##    # x #"# y
\end{lstlisting}
\end{example}

\subsubsection*{Regular Expression Literals}

\begin{bnf}
  \gd{regexstring} \eq \term{x} \ ( \ \gp{string'} \gor \gp{rawstring} \ )
\end{bnf}

A \emph{regular expression literal}\index{literal!string literal!regular expression literal} is a string literal that represents a regular expression and is compiled to \lstinline{Regex} term. The exact syntactic and semantic details of regular expressions in Intentio are implementation dependent.

\subsubsection*{Modifiers}

\begin{bnf}
  \gd{stringmod} \eq \term{t} \gor \term{u}
\end{bnf}

A \emph{string literal modifier}\index{literal!string literal!modifier} is a special flag that, when enabled, adds a step of processing of the literal value before compiling it to Intentio term. The order of modifiers is respected, they are processed from left-most modifier to right-most one.

Because modifiers alter literal contents before its compiling, they can fundamentally change their meaning, for instance the literal \lstinline{tc"   x   "} should successfully compile and evaluate to single character "\texttt{x}", while \lstinline{ct"   x   "} would report an error because there is not possible to make character from a multi-letter string.

Intentio provides following modifiers:

\begin{itemize}
  \item \texttt{t} - \emph{trim}: The trim modifier removes all white space characters from both sides of the string literal value.
  \item \texttt{u} - \emph{unindent}: The unindent modifier gets the white-space-only prefix of the string literal value and then removes it from each line of the value.
\end{itemize}

\subsubsection*{Escape Sequences}

\begin{bnf}
  \gd{charescseq} \eq \term{\textbackslash '} \gor \term{\textbackslash "} \gor \term{\textbackslash n} \gor \term{\textbackslash r} \gor \term{\textbackslash t} \gor \term{\textbackslash \textbackslash} \gor \term{\textbackslash 0} \\
  \gd{asciiescseq} \eq \term{\textbackslash x} \ \gp{hexdig} \ \gp{hexdig} \\
  \gd{unicodeescseq} \eq \term{\textbackslash u\{} \ \gmanya{ \ \gp{hexdig} \gor \term{\_} \ } \ \term{\}} \\
  \\
  \gd{escseq} \eq \gp{charescseq} \gor \gp{asciiescseq} \gor \gp{unicodeescseq}
\end{bnf}

Some \emph{escape sequences}\index{escape sequence} are available in non-raw string literals. An escape starts with a backslash character (\texttt{\textbackslash}) and continues with one of the following forms:

\begin{itemize}
  \item An \emph{8-bit code point escape sequence} starts with letter \texttt{x} and is followed by exactly two hex digits with value up to \(0x7f\). It denotes an ASCII character with value equal to provided hex value. Higher values are not permitted because it is ambiguous whether they mean Unicode code points or byte values, though the implementation should accept them on lexical level for better user experience.
  \item A \emph{24-bit code point escape sequence} starts with letter \texttt{u} and is followed by up to six hex digits surrounded by braces "\texttt{\{}" and "\texttt{\}}". It denotes the Unicode code point equal to the provided hex value. The implementation should accept more digits on lexical level for better user experience.
  \item The \emph{character escape sequences} are convenience shortcuts for 8-bit code point escape sequences. Following table describes exact translations:

    \begin{center}
    \begin{tabular}{c|c|c}
      Escape & Character & Unicode \\
      \hline
      \texttt{\textbackslash '} & \texttt{'} & U+0027 \\
      \texttt{\textbackslash "} & \texttt{"} & U+0022 \\
      \texttt{\textbackslash n} & \texttt{\textbackslash n} & U+000A \\
      \texttt{\textbackslash r} & \texttt{\textbackslash r} & U+000D \\
      \texttt{\textbackslash t} & \texttt{\textbackslash t} & U+0009 \\
      \texttt{\textbackslash \textbackslash} & \texttt{\textbackslash} & U+005C \\
      \texttt{\textbackslash 0} & NUL & U+0000 \\
    \end{tabular}
    \end{center}
\end{itemize}

\subsection{None Literal}

The keyword none evaluates to successful none term, which is a special value none of type none. It is used to make boolean result as succ none or fail none.