\chapter{Items}

This chapter describes syntax and informal semantics of Intentio items. An \emph{item}\index{item} is a component of a module. Items are uniquely named within module and are organized in flat structure.

Items are entirely determined at compile-time, remain fixed during execution, and may reside in read-only memory.

\begin{bnf}
  \gd{itemdecl} \eq \gp{fundecl} \gcomm{function}
\end{bnf}

\section{Functions}

\begin{bnf}
  \gd{fundecl} \eq \term{fun} \ \term{\gp{id}} \ \term{(} \ \gtry{\gp{funparams}} \ \term{)} \ \gp{braceblock} \\
  \gd{funparams} \eq \term{\gp{id}} \ \gmany{ \ \term{,} \ \term{\gp{id}} \ } \ \gtry{\term{,}}
\end{bnf}

A \emph{function}\index{item!function} is a named block, along with optional set of parameters. Functions are declared with the keyword \lstinline{fun}. Function returns the result of evaluating of contained block. Functions may declare a set of input variables as parameters, through which the caller passes arguments into the function. Input variables behave the same as normal variables defined within function block.

\begin{example}[Simple function]
\begin{lstlisting}
fun the_hardest_calculation(x) {
  x * 2
}
\end{lstlisting}
\end{example}
