\chapter{Items}

This chapter describes syntax and informal semantics of Intentio items. An \emph{item}\index{item} is a component of a module. Items are uniquely named within module and are organized in flat structure.

\begin{bnf}
  \gd{itemdecl} \eq    \gp{fundecl}    \gcomm{function}
                \gorln \gp{importdecl} \gcomm{import declaration}
\end{bnf}

\section{Functions}

\begin{bnf}
  \gd{fundecl} \eq \term{fun} \ \term{\gp{id}} \ \term{(} \ \gtry{\gp{funparams}} \ \term{)} \ \gp{braceblock} \\
  \gd{funparams} \eq \term{\gp{id}} \ \gmany{ \ \term{,} \ \term{\gp{id}} \ } \ \gtry{\term{,}}
\end{bnf}

A \emph{function}\index{item!function} is a named block, along with optional set of parameters. Functions are declared with the keyword \lstinline{fun}. Function returns the result of evaluating of contained block. Functions may declare a set of input variables as parameters, through which the caller passes arguments into the function. Input variables behave the same as normal variables defined within function block.

The number of function parameters $n$ is called the \emph{arity}\index{items!function arity} of the function.

\begin{example}[Simple function]
\begin{lstlisting}
fun the_hardest_calculation(x) {
  x * 2
}
\end{lstlisting}
\end{example}

\section{Import declarations}

\begin{bnf}
  \gp{importdecl} \eq \term{import} \ \gp{importspec} \\
  \\
  \gp{importspec} \eq    \term{\gp{id}} \gcomm{qualified import}
                  \gorln \term{\gp{id}} \ \term{as} \ \term{\gp{id}} \gcomm{renamed qualified import}
                  \gorln \term{\gp{id}} \ \term{:} \ \term{\gp{id}} \gcomm{item import}
                  \gorln \term{\gp{id}} \ \term{:} \ \term{\gp{id}} \ \term{as} \ \term{\gp{id}} \gcomm{renamed item import}
                  \gorln \term{\gp{id}} \ \term{:} \ \term{*} \gcomm{import-all}
\end{bnf}

An \emph{import declaration}\index{item!import declaration} creates local name binding in current scope, synonymous with specified module. Imports can import exported items directly into current scope or they can be \emph{qualified} where whole module is imported and exported items can be accessed using \emph{qualified identifier} syntax.

Import declarations support five forms:

\begin{description}
  \item [Qualified import] \hfill \\
    Binding target module to its name: \\
    \lstinline{import a}
  \item [Renamed qualified import] \hfill \\
    Binding target module to new local name: \\
    \lstinline{import a as b}
  \item [Item import] \hfill \\
    Binding exported item of target module to its name: \\
    \lstinline{import a:x}
  \item [Renamed item import] \hfill \\
    Binding exported item of target module to new local name: \\
    \lstinline{import a:x as y}
  \item [Import-all] \hfill \\
    Binding all exported items of target module to their names: \\
    \lstinline{import a:*}
\end{description}

An import declaration declares a dependency relation between the importing and imported module. Modules may be mutually recursive, though module must not import itself, directly or indirectly.

\begin{example}[Import declarations]
\begin{lstlisting}
import io
import math:sin
import math as m

fun main() {
  f = io:open("result.txt", "w");
  io:writeln(f, sin(m:pi));
}
\end{lstlisting}
\end{example}
