\chapter{Statements and expressions}

This chapter describes syntax and semantics of Intentio \emph{expressions}. Intentio is an expression language. This means that all forms of result-producing or effect-causing evaluation fall into uniform syntax category of expressions. Usually, each kind of expression can nest within each other kind of expression, and rules for evaluation of expressions involve specifying both the result produced by the expression and the order in which its sub-expressions are themselves evaluated.

A \emph{statement}\index{statement} is either an assignment or expression. Statements serve mostly to encapsulate and explicitly sequence expression evaluation, forming \emph{blocks}. Evaluating \emph{expression statement}\index{statement!expression statement} means evaluating encapsulated expression.

\begin{bnf}
  \gd{stmt} \eq    \gp{assign}  \gcomm{assignment statement}
            \gorln \gp{expr}    \gcomm{expression statement} \\
  \\
  \gd{expr} \eq    \gp{idexpr}                      \gcomm{variable or item value expression}
            \gorln \gp{literalexpr}                 \gcomm{literal expression}
            \gorln \gp{blockexpr}                   \gcomm{block expression}
            \gorln \gp{unopexpr}                    \gcomm{unary operator expression}
            \gorln \gp{binopexpr}                   \gcomm{binary operator expression}
            \gorln \term{(} \ \gp{expr} \ \term{)}  \gcomm{parenthesized expression}
            \gorln \gp{callexpr}                    \gcomm{call expression}
            \gorln \gp{whileexpr}                   \gcomm{while loop}
            \gorln \gp{ifexpr}                      \gcomm{conditional expression}
            \gorln \gp{returnexpr}                  \gcomm{return expression}
\end{bnf}

An \emph{expression}\index{expression} is an syntactic construct that \emph{evaluates} to a term. It may either \emph{succeed} or \emph{fail} resulting in a result. During the evaluation, expression may perform \emph{side effects} for example, it can mutate some state or perform execution jump. The meaning of each kind of expression dictates several things:

\begin{itemize}
  \item Whether or not to evaluate the sub-expressions when evaluating the expression
  \item The order in which to evaluate the sub-expressions
  \item How to combine the sub-expressions' results to obtain the result of the expression
\end{itemize}

\section{Assignment statement}

% TODO: lvalues/rvalues or as in Rust: place and value expressions + describe them in value expressions. 'Assignable' trait?
% https://doc.rust-lang.org/reference/expressions.html#place-expressions-and-value-expressions

\begin{bnf}
  \gd{assign} \eq \term{\gp{id}} \ \term{=} \ \gp{expr}  \gcomm{variable assignment}
\end{bnf}

An \emph{assignment statement}\index{statement!assignment statement} mutates already declared variable, or declares new variable in current scope.

\section{Variable or item value expressions}

\begin{bnf}
  \gd{idexpr} \eq \term{\gp{qid}} \gor \term{\gp{id}}
\end{bnf}

An identifier or qualified identifier used in expression context\index{expression!identifier} denotes either local variable or declared item.

\section{Literal expressions}

\begin{bnf}
  \gd{literalexpr} \eq \term{\gp{literal}}
\end{bnf}

A \emph{literal expression}\index{expression!literal expression} consists of one of the literal tokens. It directly describes a number or string.

Literals in Intentio are always immutable, it is not possible to change the value of the literal in-place.

\section{Block expressions}

\begin{bnf}
  \gd{blockexpr} \eq \gp{braceblock}  \ \gp{parenblock} \\
  \\
  \gd{braceblock} \eq \term{\{} \ \gtry{\gp{blockbody}} \ \term{\}} \\
  \gd{parenblock} \eq \term{(} \ \gtry{\gp{blockbody}} \ \term{)} \\
  \\
  \gd{blockbody} \eq \gmany{ \ \gp{stmt} \ \term{;} \ } \ \gp{expr} \ \gtry{\term{;}}
\end{bnf}

A \emph{block expression}\index{expression!block expression} is an optional sequence of statements, followed by an expression. Each block introduces a new namespace scope, this means that variables introduced within a block are in scope for only the block itself.

A parenthesized block evaluates statements sequentially while each statement evaluates successfully. On the first failing statement, the block stops evaluation and its result is equal to the result of the failing statement. Otherwise, if all statements succeeded, the block evaluates to result of the last expression. If the block is empty, its result is a succeeded \emph{none} term.

A bracked block also evaluates statements sequentially, but it does not stop evaluating statements even if some of them fails. Its result is always equal to the result of the last expression.
\section{Operator expressions}\index{expression!operator expression}

\begin{bnf}
  \gd{unopexpr} \eq \gp{negexpr} \gor \gp{notexpr} \\
  \gd{binopexpr} \eq \gp{arithexpr} \gor \gp{cmpexpr} \gor \gp{boolexpr} \\
  \\
  \gd{negexpr} \eq \term{-} \ \gp{expr} \\
  \gd{notexpr} \eq \term{not} \ \gp{expr} \\
  \\
  \gd{arithexpr} \eq    \gp{expr} \ ( \ \term{+} \gor \term{-} \ ) \ \gp{expr}
                 \gorln \gp{expr} \ ( \ \term{*} \gor \term{/} \ ) \ \gp{expr} \\
  \\
  \gd{cmpexpr} \eq \gp{expr} \ \gp{cmpop} \ \gp{expr} \\
  \gd{cmpop}   \eq \term{==} \gor \term{!=} \gor \term{<} \gor \term{>} \gor \term{<=} \gor \term{>=} \gor \term{===} \gor \term{!==} \\
  \\
  \gd{boolexpr} \eq    \gp{expr} \ \term{or} \ \gp{expr}
                \gorln \gp{expr} \ \term{and} \ \gp{expr}
\end{bnf}

\begin{bnfutils}
\begin{table}[ht]
  \caption{Operator precedence}
  \begin{center}
  \begin{tabular}[t]{c|l|c}
    \bfseries{Precedence} & \multicolumn{1}{c|}{\bfseries{Operators}} & \bfseries{Associativity} \\
    \hline
    6 & \term{-x}, \term{not x} & right-to-left \\
    5 & \term{*}, \term{/} & left-to-right \\
    4 & \term{+}, \term{-} & left-to-right \\
    3 & \term{==}, \term{!=}, \term{<}, \term{>}, \term{<=}, \term{>=}, \term{===}, \term{!==} & left-to-right \\
    2 & \term{and} & left-to-right \\
    1 & \term{or} & left-to-right
  \end{tabular}
  \end{center}
\end{table}
\end{bnfutils}

\subsection{Arithmetic operators}

Arithmetic operators apply to numeric terms (and, rarely, strings). Evaluating arithmetic operations always succeeds, behaviour in erroneous situations (such as division by zero) is described below. The following table lists all arithmetic operations, along with essential properties.

\begin{bnfutils}
\begin{table}[ht]
  \caption{Arithmetic operators}
  \begin{center}
  \begin{tabular}[t]{c|l|l}
    \bfseries{Operator} & \bfseries{Meaning} & \bfseries{Applies to types} \\
    \hline
    \term{-x} & negation & integers, floats \\
    \term{+} & sum & integers, floats, strings \\
    \term{-} & difference & integers, floats \\
    \term{*} & product & integers, floats \\
    \term{/} & quotient & integers, floats
  \end{tabular}
  \end{center}
\end{table}
\end{bnfutils}

\subsubsection{Negation}

For integer and floating-point operand terms, the \lstinline{-x} operator is defined as follows. Term type is not changed during evaluation.

\begin{lstlisting}
-x === 0 - x
\end{lstlisting}

\subsubsection{Integer and floating-point binary operations}

For \lstinline{+}, \lstinline{-}, \lstinline{*} and \lstinline{/} operators, following typing rules hold:

\begin{lstlisting}
Int + Int -> Int
Float + Float -> Float
Float + Int -> Float       # == Float + Float(Int)
Int + Float -> Float       # == Float(Int) + Float
\end{lstlisting}

\subsubsection{String concatenation}

Strings can be concatenated using \lstinline{+} operator. Following typing rules hold:

\begin{lstlisting}
Char + Char -> String      # concatenate two Chars making a String
String + String -> String  # concatenate two Strings
Char + String -> String    # prepend Char to String
String + Char -> String    # append Char to String
\end{lstlisting}

\subsection{Term comparison operators}

Comparison operators compare two operands and yield either success if comparison passed, or failure.

\begin{bnfutils}
\begin{table}[ht]
  \caption{Term comparison operators}
  \begin{center}
  \begin{tabular}[t]{c|l}
    \bfseries{Operator} & \bfseries{Meaning} \\
    \hline
    \term{==} & equal to \\
    \term{!=} & not equal to \\
    \term{<} & less than \\
    \term{>} & greater than \\
    \term{<=} & less than or equal to \\
    \term{>=} & greater than or equal to \\
    \term{===} & strict equal to \\
    \term{!==} & strict not equal to
  \end{tabular}
  \end{center}
\end{table}
\end{bnfutils}

In Intentio, arguments of comparison operators can be of any type. The following total ordering among types is defined:

\begin{lstlisting}
Number (Int or Float) < Char < String < Regex
\end{lstlisting}

Integers are comparable with integers in the usual way. Floating-point numbers are comparable with floating-point numbers according to IEEE-754\cite{IEEE754} rules. For \lstinline{===} and \lstinline{!==} operators comparing integer to floating-point number (and vice-versa) always fails, otherwise integer is converted to float before comparison.

Characters are compared using their byte representation. Strings and regular expressions are compared character by character. For \lstinline{===} and \lstinline{!==} operators comparisons between characters, strings and regular expressions always fail, otherwise characters and regular expressions are always converted to strings before \lstinline{==} and \lstinline{!=} comparison.

No matter what comparison was made, its result term is equal to second argument's term.

\subsection{Result operators}

Result operators are used to making logical expressions, such as conjunction, alternative, xor and negation. Their actions are analogous to those from which they took their name in classical mathematics.

The following tables present the exact results for each set of function arguments. The descriptions in the left column correspond to the first argument, while those in the top column - to the second. The results term is copied from the second argument.

\begin{bnfutils}
  \begin{table}[ht]
    \caption{\emph{and}}
    \begin{center}
    \begin{tabular}[t]{l|l|l}
       & Succ Term2 & Fail Term2 \\
      \hline
      Succ Term1 & Succ Term2 & Fail Term2 \\
      Fail Term1 & Fail Term2 & Fail Term2 
    \end{tabular}
    \end{center}
  \end{table}
\end{bnfutils}

\begin{bnfutils}
  \begin{table}[ht]
    \caption{\emph{or}}
    \begin{center}
    \begin{tabular}[t]{l|l|l}
        & Succ Term2 & Fail Term2 \\
      \hline
      Succ Term1 & Succ Term2 & Succ Term2 \\
      Fail Term1 & Succ Term2 & Fail Term2 
    \end{tabular}
    \end{center}
  \end{table}
\end{bnfutils}

\begin{bnfutils}
  \begin{table}[ht]
    \caption{\emph{xor}}
    \begin{center}
    \begin{tabular}[t]{l|l|l}
        & Succ Term2 & Fail Term2 \\
      \hline
      Succ Term1 & Fail Term2 & Succ Term2 \\
      Fail Term1 & Succ Term2 & Fail Term2 
    \end{tabular}
    \end{center}
  \end{table}
\end{bnfutils}

\begin{bnfutils}
  \begin{table}[!htbp]
    \caption{\emph{not}}
    \begin{center}
    \begin{tabular}[t]{l|l}
      Succ Term & Succ Term \\
      Fail Term & Fail Term 
    \end{tabular}
    \end{center}
  \end{table}
\end{bnfutils}

\subsection{Typing errors}

Not all typing rules are defined, and hence using undefined combinations is prohibited. Evaluating such expressions results in run-time type error panic.

Example erroneous situations: \lstinline{Regex + Regex}, \lstinline{String + Regex}.

\subsection{Division by zero}

For integer operands, division by zero results in run-time panic. The result of a floating-point division by zero is not specified beyond the IEEE-754\cite{IEEE754} standard; whether a run-time panic occurs is implementation-specific.

\subsection{Integer overflow}

Operators \lstinline{+}, \lstinline{-}, \lstinline{*} and \lstinline{/} may legally overflow, resulting value exists and is deterministically defined by the signed integer representation on destination architecture, the operation, and its operands. Evaluation is not failing nor no panic is triggered as a result of overflow. Intentio compiler may not optimize code under the assumption that overflow does not occur.

\section{Call expressions}

\begin{bnf}
  \gd{callexpr} \eq \gp{expr} \ \term{(} \ \gtry{\gp{callargs}} \ \term{)} \\
  \gd{callargs} \eq \gp{expr} \ \gmany{ \ \term{,} \ \gp{expr} \ } \ \gtry{\term{,}}
\end{bnf}

A \emph{call expression}\index{expression!call expression} is made of an expression followed by a parenthesized, comma-separated expression list. It invokes a function, providing zero or more arguments, evaluating to result of function invocation (either succeeded return term or failure). If evaluation of any of function arguments fails, then the function is not called, and instead whole expression yields argument failure.

\begin{example}[Trivial call expression]
\begin{lstlisting}
y = add(3, 4);
\end{lstlisting}
\end{example}

\begin{example}[Calling arbitrary expressions]
\begin{lstlisting}[mathescape=true]
fun f(x) { x * 2; }
fun g() { f; }

y = g()(5); # works same as y = f(5)
y == 10;
\end{lstlisting}
\end{example}

\section{Loops}

\begin{bnf}
  \gd{whileexpr} \eq \term{while} \ ( \ \gp{parenblock} \gor \gp{expr} \ ) \ \gp{braceblock}
\end{bnf}

A \emph{while loop}\index{expression!while loop} begins by evaluating loop condition block/expression. If evaluating loop condition succeeds, the loop body block executes. If the condition fails, then while loop completes and returns succeeded \emph{none} term. The result value of condition block is ignored.

\begin{example}[While expression with parentheses]
\begin{lstlisting}
  i = 0;
  while (i < 8; i = i + 1) {
    println(i);
  }
\end{lstlisting}
\end{example}

\section{Conditionals}

\begin{bnf}
  \gd{ifexpr} \eq \term{if} \ ( \ \gp{parenblock} \gor \gp{expr} \ ) \ \gp{braceblock} \\
              & & \gtry{ \ \term{else} \ ( \ \gp{braceblock} \gor \gp{ifexpr} \ ) \ }
\end{bnf}

A \emph{conditional expression}\index{expression!conditional expression} is a conditional branch in program control. A conditional expression is made of a condition block/expression, followed by a consequent block, any number of alternative conditions and blocks, and an optional trailing final alternative block. If a condition expression evaluates successfully, the consequent block is executed and any subsequent \lstinline{else if} or \lstinline{else} block is skipped. If a condition expression evaluation fails, the consequent block is skipped and subsequent \lstinline{else if} or \lstinline{else} condition is evaluated. If all condition blocks fail then \lstinline{else} block is executed. Result values of conditions in condition block are ignored. A conditional expression evaluates to the same value as the executed block, or succeeded \emph{none} if no block is evaluated.

\begin{example}[Trivial if expression]
\begin{lstlisting}
  if a < b { c = a };
\end{lstlisting}
\end{example}

The variables defined in the condition could be used in the if-braceblock and the else-braceblock as it was the same scope.

\begin{example}[If expression]
  \begin{lstlisting}
    if (a = 3; c = b; a < b) { c = a } 
    else { println("c is equal to b which value is " + b) };
  \end{lstlisting}
\end{example}

\section{Return expressions}

\begin{bnf}
  \gd{returnexpr} \eq \term{return} \ \gtry{\gp{expr}}
\end{bnf}

A \emph{return expression}\index{expression!return expression} terminates execution of enclosing function, making it return the result of evaluating provided expression. Just in case no expression is given, a \emph{none result} is returned.

\begin{example}[Return expression]
\begin{lstlisting}
fun abs(a) {
  if a < 0 {
    return -a;
  }
  return a;
}
\end{lstlisting}
\end{example}
