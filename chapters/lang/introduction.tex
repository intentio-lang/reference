\chapter{Introduction}

Intentio is domain specific, imperative programming language oriented for processing textual data. Intentio provides goal-oriented execution, generators, strong dynamic typing with optional type annotations, and a rich set of primitive data types, including Unicode strings, lists, arrays, maps, sets, arbitrary and fixed precision integers, and floating-point numbers. Intentio tries to incorporate ideas of the Icon\cite{TheIconProgrammingLanguage} programming language into modern programming patterns.

This part defines the syntax of Intentio language and informal abstract semantics for the meaning of such programs. We leave as implementation detail how Intentio programs are manipulated, interpreted, compiled, etc. This includes all steps from source code to running program, programming environments and error messages. This also means that this document do not describe the reference compiler for Intentio language - \emph{intentioc}\cite{intentioc}.

\section{Notation}

Throughout this document a BNF-based notational syntax is used to describe lexical structure and grammar:

\[
nonterminal ::= \texttt{term} | \texttt{another} \texttt{term}+
nonterm2    ::= nonterminal*
\]

Following conventions are used for presenting productions syntax:

\begin{center}
  \begin{tabular}[t]{rl}
    \((pattern)\)                & grouping \\
    \([pattern]\)                & optional \\
    \(pattern*\)                 & zero or more repetitions \\
    \(pattern+\)                 & one or more repetitions \\
    \(pattern_a | pattern_b\)    & choice \\
    \(\texttt{token}\)           & terminal symbol (in fixed-width font)
  \end{tabular}
\end{center}

\subsubsection{Unicode productions}

A few productions in Intentio's grammar permit Unicode\cite{Unicode6} code points outside the ASCII range. These productions are defined in terms of character properties specified in the Unicode standard, rather than in terms of ASCII-range code points. Intentio compilers are expected to make use of new versions of Unicode as they are made available.


\section{Program structure}

% TODO:


\section{Values and types}

% TODO:


\section{Namespaces}

% TODO:
